\section{Related Work}

The concept of live programming has been used to build tools for continuous feedback in a number of languages, but these have mostly been for visual languages such as Morphic [5] and PureData [7], or languages designed specifically for live programming such as YinYang [6].
While some of these tools like YinYang are able to provide more information about the program being executed through built-in language features, the language itself is not commonly used.
The overarching cooperative programming project that Exampler is a part of aims to support Haskell, a language prevalent throughout academia and industry.

Specific to Haskell, there is a tool called QuickCheck [2] that tries to check specific properties of a function by testing if the property holds for a large volume of random examples.
While this is useful for testing once the code is written, it generates a massive number of examples and is not easily processed and understood by the programmer.
Even so, QuickCheck is still quite useful and some of its modules are leveraged by Exampler to generate random examples.

Finally, there are many tools and IDEs that will automatically run tests for a program as the programmer updates it.
These include Guard for Ruby and NCrunch for the Visual Studio IDE.
While these are certainly handy, they require the programmer to write tests beforehand instead of giving automatic feedback.
Also, since the tests are manually entered, this approach does not work when there are major structural changes (i.e., type signature changes) to the code base.




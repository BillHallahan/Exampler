\section{approach}

This project creates a tool called Exampler that takes as input a Haskell source file and the name of a function within that source file.
Exampler will generate a set of representative examples for the specified function and watch the source file for changes to know when to update the set of examples.
	
We define a set of representative examples as a set of examples that provides significant or complete branch coverage of the specified function.
The program attempts to generate such a set by repeatedly testing the function on random examples and then observing differences in branch coverage.
If an example improves branch coverage, then it is kept as part of the representative set.
Given enough random examples, it is possible to get full branch coverage of all reachable branches in a program.
Since only examples that improve branch coverage are included in the set, we have also constructed an example set that is small enough for the programmer to understand.

As for example reuse, it appears that it would be as simple as keeping the generated examples in memory.
The issue is that Haskell is a compiled language, so the program would need to be recompiled whenever the source file of interest is modified.
The obvious solution is to write the examples to a file, but this does not work for examples for higher-order functions since first-order functions have no textual representation in the compiled program.
The only way to deal with this is to dynamically reload the module as it changes so the program can keep all of the examples in memory.

Concerning polymorphic functions, the program inspects the type structure of the function of interest and builds a new type by replacing the type variables with concrete types that are valid given the function's context.
The concrete type can then be used to generate examples for the polymorphic function.

All of these features result in a robust tool that can generate sets of representative examples for a function and maintain them as the function changes.
Furthermore, the program is just a simple executable wrapper around a reusable module that can be leveraged in any cooperative programming or live programming environment.


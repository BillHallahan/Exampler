\iffalse

\section*{Cooperative Programming: Integrating Software Synthesis with the Live Paradigm}

\paragraph{Overview:} Live programming is an emerging paradigm that is promising a vast change in the techniques used to develop modern software. A live programming environment allows a programmer to immediately see the effects of changes to a program on its outputs and effectively eliminates the edit-run-debug cycle that dominates programming workflows today. Our work seeks to develop new synthesis techniques that make use of the real-time feedback loop provided by a live programming environment. We will develop a system that allows a user to track a set of examples and synthesize subroutines to fit that set. When combined with fault localization techniques, programmers will be able to quickly find incorrect sections of code and initiate repairs that leverage information gleaned from the provided examples. While current  development tools provide tangible benefits to programmers skilled enough to use them, we aim to develop an accessible, first-of-its-kind, \textit{cooperative programming} environment that provides feedback in the form of generating representative examples and allowing changes to the code to be made through adjustments to these examples.

\paragraph{Intellectual Merits:} There are several open-ended problems to explore, including:

\begin{itemize}
\item Create novel, real-time algorithms to synthesize code within a live programming environment.
\item Investigate methods to evaluate the quality of an example set and methods to produce high quality example sets for a given program.
\item Build a formal theoretical framework for synthesis in a feedback loop.
\item Integrate these developments in a unified environment for a modern, major programming language - namely, Haskell.
\end{itemize}

\paragraph{Broader Impact:} 

\fi


We believe that cooperative programming environments will increase programmer productivity while simultaneously lowering the barriers for entry for newcomers to computer programming. Approaches to such environments must be informed by the needs of programmers, for whom these systems are ultimately built.
